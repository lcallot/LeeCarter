\documentclass[11pt,oneside, a4paper]{amsart}\usepackage[]{graphicx}\usepackage[]{color}
%% maxwidth is the original width if it is less than linewidth
%% otherwise use linewidth (to make sure the graphics do not exceed the margin)
\makeatletter
\def\maxwidth{ %
  \ifdim\Gin@nat@width>\linewidth
    \linewidth
  \else
    \Gin@nat@width
  \fi
}
\makeatother

\definecolor{fgcolor}{rgb}{0.345, 0.345, 0.345}
\newcommand{\hlnum}[1]{\textcolor[rgb]{0.686,0.059,0.569}{#1}}%
\newcommand{\hlstr}[1]{\textcolor[rgb]{0.192,0.494,0.8}{#1}}%
\newcommand{\hlcom}[1]{\textcolor[rgb]{0.678,0.584,0.686}{\textit{#1}}}%
\newcommand{\hlopt}[1]{\textcolor[rgb]{0,0,0}{#1}}%
\newcommand{\hlstd}[1]{\textcolor[rgb]{0.345,0.345,0.345}{#1}}%
\newcommand{\hlkwa}[1]{\textcolor[rgb]{0.161,0.373,0.58}{\textbf{#1}}}%
\newcommand{\hlkwb}[1]{\textcolor[rgb]{0.69,0.353,0.396}{#1}}%
\newcommand{\hlkwc}[1]{\textcolor[rgb]{0.333,0.667,0.333}{#1}}%
\newcommand{\hlkwd}[1]{\textcolor[rgb]{0.737,0.353,0.396}{\textbf{#1}}}%

\usepackage{framed}
\makeatletter
\newenvironment{kframe}{%
 \def\at@end@of@kframe{}%
 \ifinner\ifhmode%
  \def\at@end@of@kframe{\end{minipage}}%
  \begin{minipage}{\columnwidth}%
 \fi\fi%
 \def\FrameCommand##1{\hskip\@totalleftmargin \hskip-\fboxsep
 \colorbox{shadecolor}{##1}\hskip-\fboxsep
     % There is no \\@totalrightmargin, so:
     \hskip-\linewidth \hskip-\@totalleftmargin \hskip\columnwidth}%
 \MakeFramed {\advance\hsize-\width
   \@totalleftmargin\z@ \linewidth\hsize
   \@setminipage}}%
 {\par\unskip\endMakeFramed%
 \at@end@of@kframe}
\makeatother

\definecolor{shadecolor}{rgb}{.97, .97, .97}
\definecolor{messagecolor}{rgb}{0, 0, 0}
\definecolor{warningcolor}{rgb}{1, 0, 1}
\definecolor{errorcolor}{rgb}{1, 0, 0}
\newenvironment{knitrout}{}{} % an empty environment to be redefined in TeX

\usepackage{alltt}
\usepackage{natbib}

\usepackage{amsbsy,amsmath}
\usepackage{amssymb,amsfonts}
\usepackage{booktabs,url,enumerate}
\usepackage{color,xcolor}
\usepackage{float}
\usepackage{tikz}
\usepackage{rotating,graphicx,lscape}
\usepackage{commath}
\usetikzlibrary{arrows,positioning} 
\usepackage[hypcap]{caption}
\usepackage{setspace}



\usepackage[top=1.5cm, bottom=1.5cm, left=3.0cm, right=3.0cm]{geometry}
\IfFileExists{upquote.sty}{\usepackage{upquote}}{}
\begin{document}

\title{Generating log mortality R binary files}   
\author{LC NH MKL}
\date{\today}
\maketitle


This document is used to extract the mortality rates from the raw data of the HDM, and saves R objects containing the log mortality rates. 







We first define the samples.
\begin{knitrout}
\definecolor{shadecolor}{rgb}{0.969, 0.969, 0.969}\color{fgcolor}\begin{kframe}
\begin{alltt}
\hlcom{# Defining the sample}
\hlstd{smpl} \hlkwb{=} \hlkwd{list}\hlstd{(}\hlkwc{cn} \hlstd{=} \hlkwa{NULL}\hlstd{,} \hlkwc{gen} \hlstd{=} \hlkwa{NULL}\hlstd{,} \hlkwc{startyear} \hlstd{=} \hlnum{1900}\hlstd{,} \hlkwc{endyear} \hlstd{=} \hlnum{2010}\hlstd{,} \hlkwc{minage} \hlstd{=} \hlnum{0}\hlstd{,}
    \hlkwc{maxage} \hlstd{=} \hlnum{90}\hlstd{)}

\hlstd{cnall} \hlkwb{<-} \hlkwd{c}\hlstd{(}\hlstr{"FRA"}\hlstd{)}
\end{alltt}
\end{kframe}
\end{knitrout}



Then we loop over every combination of country and gender ($+$ total). 
\begin{knitrout}
\definecolor{shadecolor}{rgb}{0.969, 0.969, 0.969}\color{fgcolor}\begin{kframe}
\begin{alltt}
\hlstd{hdmpath} \hlkwb{<-} \hlkwd{paste0}\hlstd{(}\hlkwd{getwd}\hlstd{(),} \hlstr{"/HMD"}\hlstd{)}

\hlkwa{for} \hlstd{(cn} \hlkwa{in} \hlstd{cnall) \{}
    \hlkwa{for} \hlstd{(gen} \hlkwa{in} \hlkwd{c}\hlstd{(}\hlstr{"female"}\hlstd{,} \hlstr{"male"}\hlstd{,} \hlstr{"total"}\hlstd{)) \{}
        \hlstd{smpl}\hlopt{$}\hlstd{cn} \hlkwb{<-} \hlstd{cn}
        \hlstd{smpl}\hlopt{$}\hlstd{gen} \hlkwb{<-} \hlstd{gen}

        \hlkwd{HDM2logm}\hlstd{(smpl, hdmpath)}
    \hlstd{\}}
\hlstd{\}}
\end{alltt}
\end{kframe}
\end{knitrout}


The log mortality rates are now saved as R objects named country\_ gender.
\end{document}
